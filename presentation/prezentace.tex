\documentclass{beamer}


%\usepackage{multimedia}
\usepackage[czech]{babel}
%\usepackage{enumitem}

%\usetheme{Boadilla}
\usetheme{default}
\usecolortheme{seahorse}
\definecolor{myblue}{rgb}{0.2, 0.2, 0.6}
\setbeamertemplate{caption}{{\color{myblue}Obrázek:} \raggedright\insertcaption\par}
\setbeamertemplate{footline}[frame number] 

\newcommand{\authorname}{Matěj Dluhoš, Jan Gaja, Petr Gajdošík, Jan Rádl, Petr Jurásek, Phan Thanh Tu}
\newcommand{\authorsshort}{Dluhoš, Gaja, Gajdošík, Rádl, Jurásek, Phan}
\newcommand{\thesisname}{PDS: Python}

\title{\thesisname}
\author{\authorname}
\institute{Univerzita Palackého v Olomouci}
\date{\today}

\setbeamertemplate{navigation symbols}{}
\setbeamertemplate{headline}{}
\setbeamertemplate{footline}{
  \leavevmode%
  \hbox{%
  \begin{beamercolorbox}[wd=.4\paperwidth,ht=2.25ex,dp=1ex,center]{author in head/foot}%
    \usebeamerfont{author in head/foot}\authorsshort
  \end{beamercolorbox}%
  \begin{beamercolorbox}[wd=.3\paperwidth,ht=2.25ex,dp=1ex,center]{title in head/foot}%
    \usebeamerfont{title in head/foot}\thesisname
  \end{beamercolorbox}%
  \begin{beamercolorbox}[wd=.3\paperwidth,ht=2.25ex,dp=1ex,right]{date in head/foot}%
    \usebeamerfont{date in head/foot}\insertshortdate{}\hspace*{2em}
    \insertframenumber{} / \inserttotalframenumber\hspace*{2ex} 
  \end{beamercolorbox}}%
  \vskip0pt%
}

%\logo{\includegraphics[height=1cm]{UP_logo.png}}
\usepackage{tikz}
\addtobeamertemplate{headline}{}{%
    \begin{tikzpicture}[overlay, remember picture]
    	\ifnum\insertframenumber>1
        	\node[anchor=north east, inner sep=5pt] at (current page.north east) {\includegraphics[height=1cm]{obrazky/UP_logo.png}};
        \fi
    \end{tikzpicture}
}

\setbeamercolor{titlelike}{bg=,fg=}

\begin{document}

\begin{frame}
	\titlepage
\end{frame}

\begin{frame}{Obsah}
	\tableofcontents
\end{frame}

\section{Představení bakalářské práce}
\subsection{Cíle práce}
\begin{frame}{Cíle práce}
	\begin{itemize}
		\item \textbf{Porovnat statický a dynamický jazyk pro programování mikrokontrolérů.}
		\item \textbf{Snadnost implementace řešení v zastupujících jazycích.}
			\begin{itemize}
				\item[\textendash] Jednoduchost vytvoření nového projektu.
				\item[\textendash] Snadnost experimentování.
				\item[\textendash] Budoucí rozvoj aplikace.
			\end{itemize}
		\item \textbf{Porovnání výkonu zástupců statického a dynamického jazyka.}
			\begin{itemize}
				\item[\textendash] Rychlost.
				\item[\textendash] Paměťová náročnost.
			\end{itemize}
	\end{itemize}
\end{frame}

\subsection{Důvod zvolení práce}
\begin{frame}{Důvod zvolení práce}
	\begin{itemize}
		%\item \textbf{Výběr vhodného programovacího jazyka může být složitý proces.}
		\item \textbf{Výběr vhodného programovacího jazyka může být složitý.}
		\item \textbf{Rostoucí význam mikrokontrolérů.}
			\begin{itemize}
				\item[\textendash] S rozvojem IoT se zvyšuje důraz na vývoj efektivních a spolehlivých vestavěných systémů.
			\end{itemize}
		\item \textbf{Nová zkušenost.}
	\end{itemize}

\end{frame}

\section{Rozbor bakalářské práce}
\subsection{Co to je statický a dynamický proces?}
\begin{frame}{Co je to statický a dynamický proces?}
	\begin{itemize}
		\item \textbf{Statický proces}
			\begin{itemize}
				\item[\textendash] Vlastnosti nebo konfigurace se nemění.
				\item[\textendash] Veškeré typové kontroly a alokace paměti jsou provedeny při kompilaci.
				\item[\textendash] Větší předvídatelnost a často i lepší výkon.
			\end{itemize}
		\item \textbf{Dynamický proces}
			\begin{itemize}
				\item[\textendash] Schopnost měnit své chování nebo konfiguraci v reálném čase.
				\item[\textendash] Typy proměnných mohou být určeny a změněny za běhu programu.
				\item[\textendash] Umožňuje větší flexibilitu a adaptabilitu.
				\item[\textendash] Může mít následky na výkon a bezpečnost.
			\end{itemize}
	\end{itemize}

\end{frame}

\section{Test}
\subsection{Co to je?}
\begin{frame}{test8}
...
\end{frame}

\end{document}